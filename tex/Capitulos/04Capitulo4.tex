%---------------------------------------------------------------------
%
%                          Cap�tulo 2
%
%---------------------------------------------------------------------

\chapter{Seed Cases and ELO Tournament}

\begin{resumen}
In this chapter we outline and explain the our techniques used to ensure new users and domains have a valid starting point from which users can productively use the tool.
\end{resumen}


%-------------------------------------------------------------------
\section{Motivation}
%-------------------------------------------------------------------
\label{cap1:sec:motivation}

Our objective for this chapter will be to provide details on the process chosen to develop a solid knowledge basis on which we can build a use case for our program.
What we mean by this basis is the knowledge of which metrics to use to provide a clear picture of the datasets belonging to the domain, as well as which metrics are relevant to a profile and how should they be shown, as with a graph, through a text report, which colors, etc.
We will be using the input of an expert to determine the metrics to use, and will then generate a seed profile based on an ELO tournament with a pool of experts. This will be then affected by user input.

%-------------------------------------------------------------------
\section{Metric Seeds}
%-------------------------------------------------------------------
\label{cap1:sec:motivation}

Our objective for this chapter will be to provide details on the process chosen to develop a solid knowledge basis on which we can build a use case for our program.
What we mean by this basis is the knowledge of which metrics to use to provide a clear picture of the datasets belonging to the domain, as well as which metrics are relevant to a profile and how should they be shown, as with a graph, through a text report, which colors, etc.
We will be using the input of an expert to determine the metrics to use, and will then generate a seed profile based on an ELO tournament with a pool of experts

%-------------------------------------------------------------------
\section*{\NotasBibliograficas}
%-------------------------------------------------------------------
\TocNotasBibliograficas

These are the bibliographical notes
\citep{ldesc2e}

\medskip

%Y tambi�n ponemos el acr�nimo \ac{CVS} para que no cruja.


%-------------------------------------------------------------------
\section*{\ProximoCapitulo}
%-------------------------------------------------------------------
\TocProximoCapitulo

This is the next chapter section

% Variable local para emacs, para  que encuentre el fichero maestro de
% compilaci�n y funcionen mejor algunas teclas r�pidas de AucTeX
%%%
%%% Local Variables:
%%% mode: latex
%%% TeX-master: "../Tesis.tex"
%%% End:
