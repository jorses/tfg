%---------------------------------------------------------------------
%
%                          Cap�tulo 2
%
%---------------------------------------------------------------------

\chapter{Seed Cases and ELO Tournament}

\begin{resumen}
In this chapter we outline and explain the our techniques used to ensure new users and domains have a valid starting point from which users can productively use the tool.
\end{resumen}


%-------------------------------------------------------------------
\section{Motivation}
%-------------------------------------------------------------------
\label{cap1:sec:motivation}

Our objective for this chapter will be to provide details on the process chosen to develop a solid knowledge basis on which we can build a use case for our program.

What we mean by this basis is the knowledge of which metrics to use to provide a clear picture of the datasets belonging to the domain, as well as which metrics are relevant to a profile and how should they be shown, as with a graph, through a text report, which colors, etc.

We will be using the input of an expert to determine the metrics to use, and will then generate a seed profile based on an ELO tournament with a pool of experts. This will be then affected by user input.

%-------------------------------------------------------------------
\section{Case Seeds}
%-------------------------------------------------------------------
\label{cap1:sec:metricseeds}

The process to establish a frame of knowledge for a certain domain will be initiated by an expert who will provide its associated class, 
or basic categorization of users, and a brief computer-understood description of what he's interested in.

This knowledge will be used to run a first analysis and generate an array of reports through a semi randomization process, 
which will be then pooled together as participants of a tournament, and given an initial ELO of 1000.

The final winner of this tournament will be taken as the "seed" for the classification the users belonged to, 
so if a new user enters the program and belongs to this class, it will be compared to people of this class through a metric and given the report of the person that is closer to them.

The attributes of the user that form the metric are different for each class, and can be things like gender, age, medical specialty, etc.
%-------------------------------------------------------------------
\section{Experiment Design}
%-------------------------------------------------------------------
\label{cap1:sec:experimentdesign}
To test this approach to seed generation we have created an

To design the experiment we have used anonymized historical grades data from UCM's Computer Science.

Our dataset contains data about anonymized global degree grades from the nineties, containing the year of graduation and the gender among other variables.

This will be our domain.

This dataset is to be viewed from three different perspectives : 
\begin{enumerate}
\item Students wishing to know how their grade stands among their peers, 
\item Teachers who want to know how the year they've teached has faired compared to the others, 
\item The figure of a gender delegate who wishes to know if the grade distributions are different when broken down by gender.
\end{enumerate}

Our objective first is to generate successful seed cases for each class which will then be used as base report generation techniques for each category of user within this domain. 
%-------------------------------------------------------------------
\section{Experiment results}
%-------------------------------------------------------------------
\label{cap1:sec:experimentresults}

