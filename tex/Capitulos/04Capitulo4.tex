%---------------------------------------------------------------------
%
%                          Cap�tulo 2
%
%---------------------------------------------------------------------

\chapter{Seed Cases and ELO Tournament}

\begin{resumen}
In this chapter we outline and explain the our techniques used to ensure new users and domains have a valid starting point from which users can productively use the tool.
\end{resumen}

\linespread{1.6}

%-------------------------------------------------------------------
\section{Motivation}
%-------------------------------------------------------------------
\label{cap4:sec:motivation}

Our objective for this chapter will be to provide details on the process chosen to develop a solid knowledge basis on which we can build a use case for our program.

What we mean by this basis is the knowledge of which metrics to use to provide a clear picture of the datasets belonging to the domain, as well as which metrics are relevant to a profile and how should they be shown, as with a graph, through a text report, which colors, etc.

We will be using the input of an expert to determine the metrics to use, and will then generate a seed profile based on an ELO tournament with a pool of experts. This will be then affected by user input.

%-------------------------------------------------------------------
\section{Case Seeds}
%-------------------------------------------------------------------
\label{cap4:sec:metricseeds}

The process to establish a frame of knowledge for a certain domain will be initiated by an expert who will provide its associated class, 
or basic categorization of users, and a brief computer-understood description of what he's interested in.

This knowledge will be used to run a first analysis and generate an array of reports through a semi randomization process, 
which will be then pooled together as participants of a tournament, and given an initial ELO of 1000.

The final winner of this tournament will be taken as the "seed" for the classification the users belonged to, 
so if a new user enters the program and belongs to this class, it will be compared to people of this class through a metric and given the report of the person that is closer to them.

The attributes of the user that form the metric are different for each class, and can be things like gender, age, medical specialty, etc.
%-------------------------------------------------------------------
\section{Experiment Design}
%-------------------------------------------------------------------
\label{cap4:sec:experimentdesign}
To test this approach to seed generation we have created an

To design the experiment we have used anonymized historical grades data from UCM's Computer Science.

Our dataset contains data about anonymized global degree grades from the nineties, containing the year of graduation and the gender among other variables.

This will be our domain.

This dataset is to be viewed from three different perspectives : 
\begin{enumerate}
\item Students wishing to know how their grade stands among their peers, 
\item Teachers who want to know how the year they've teached has faired compared to the others, 
\item The figure of a gender delegate who wishes to know if the grade distributions are different when broken down by gender.
\end{enumerate}

Our objective first is to generate successful seed cases for each class which will then be used as base report generation techniques for each category of user within this domain. 
%-------------------------------------------------------------------
\section{The Elo Rating System}
%-------------------------------------------------------------------
\label{cap4:sec:elorating}

\begin{figure}[!htb]
    \center{\includegraphics[width=\textwidth]
    {Capitulos/uscf.png}}
    \caption{\label{fig:my-label2} ELO for US Chess Federation Ratings  [4.1]}
\end{figure}

The Elo rating system is a method for estimating the skill levels of players relative to each other used in zero sum games, most famously chess. It has been adapted to numerous other ratings for specific use cases and situations.[4.2]

The difference between the scores of two different players serves as a good predictor of the match results.

In our example, the "players" will be the reports generated by our seed generation system, and the winner of the match will be the report chosen by the expert users through the click of a button.

A reports's Elo rating is represented by a number, the higher the better.

We've made our initial rating for every report before initializing the tournament as 1000.

The rating goes up or down based on the result of different matches between reports. 

After every game, the winner takes rating away from the loser and onto himself.
The difference between the ratings of the winner and loser makes the result vary in quantity, so if a player with a low score wins against a player with a high score the loss is greater for the loser and the gains greater for the winner than in the other situation where the higher rated player wins.
The distance between ratings is also taken into account in this situation, the greater the difference between ratings the bigger the actual gains or loses will be.

This system is thus self correcting and expected to provide a good framework for comparing our reports.

%-------------------------------------------------------------------
\section{The Tournament}
%-------------------------------------------------------------------
\label{cap4:sec:tourney}

Our tournament is played in a Round Robin style to ensure that all reports are compared against each other.

We also have to note that the reports are distinguished between one another in different degrees, that is, two can only be differenced by the colors while the difference between two others in a match might entail content.

All in all, we're trying to make very little difference in treatment between them before presenting them to the users, because we assume that our generating system doesn't know the importance of the different aspects of the reports and might take away important information if we make it have more filtering capabilities before the tournament starts.


Elo systems tend to create distributions such as the one in Figure 4.1, giving us a clear winner under normal conditions.

Our first sample for the tournament experiment will be of 12 images, thus creating enough matches for our ELO system to be effective while being small enough not to bother our user pool of experts who will essentially decide the winners of every match

%-------------------------------------------------------------------
\section{Experiment results}
%-------------------------------------------------------------------
\label{cap4:sec:experimentresults}

%-------------------------------------------------------------------
\section*{\NotasBibliograficas}
%-------------------------------------------------------------------
\TocNotasBibliograficas
\begin{itemize}
\item [4.1] ELO Distributions for the USCF https://chess.stackexchange.com/questions/2550/whats-the-average-elo-rating-whats-the-average-uscf-rating 
\item [4.2] Approximating formulas for the USCF rating system http://math.bu.edu/people/mg/ratings/approx/approx.html
\end{itemize}
