%---------------------------------------------------------------------
%
%                          Cap�tulo 2
%
%---------------------------------------------------------------------

\chapter{Proof Of Concept \\Medical Data}

\begin{resumen}
In this chapter we see all of the pieces of our program come together and work in the analysis of a set of medical data.
We will follow a step by step execution from the different user viewpoints and the program's viewpoint.
\end{resumen}

%-------------------------------------------------------------------
\section{introduction}
%-------------------------------------------------------------------
\label{cap1:sec:introduccion}

This is the introductory texttt

%-------------------------------------------------------------------
\section*{\NotasBibliograficas}
%-------------------------------------------------------------------
\TocNotasBibliograficas

These are the bibliographical notes
\citep{ldesc2e}

\medskip

%Y tambi�n ponemos el acr�nimo \ac{CVS} para que no cruja.


%-------------------------------------------------------------------
\section*{\ProximoCapitulo}
%-------------------------------------------------------------------
\TocProximoCapitulo

This is the next chapter section

% Variable local para emacs, para  que encuentre el fichero maestro de
% compilaci�n y funcionen mejor algunas teclas r�pidas de AucTeX
%%%
%%% Local Variables:
%%% mode: latex
%%% TeX-master: "../Tesis.tex"
%%% End:
