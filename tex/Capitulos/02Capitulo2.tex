%---------------------------------------------------------------------
%
%                          Cap�tulo 2
%
%---------------------------------------------------------------------

\chapter{The program structure}

\begin{resumen}
In this chapter we outline the logical structure of the program developed, its different modules, functions and technologies employed to make them.
We provide both a high level abstraction and a low level code representation for each class.
\end{resumen}


%-------------------------------------------------------------------
\section{Introduction}
%-------------------------------------------------------------------
\label{cap1:sec:introduccion}

This is the introductory texttt

%-------------------------------------------------------------------
\section*{\NotasBibliograficas}
%-------------------------------------------------------------------
\TocNotasBibliograficas

These are the bibliographical notes
\citep{ldesc2e}

\medskip

%Y tambi�n ponemos el acr�nimo \ac{CVS} para que no cruja.


%-------------------------------------------------------------------
\section*{\ProximoCapitulo}
%-------------------------------------------------------------------
\TocProximoCapitulo

This is the next chapter section

% Variable local para emacs, para  que encuentre el fichero maestro de
% compilaci�n y funcionen mejor algunas teclas r�pidas de AucTeX
%%%
%%% Local Variables:
%%% mode: latex
%%% TeX-master: "../Tesis.tex"
%%% End:
