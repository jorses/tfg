%---------------------------------------------------------------------
%
%                          Cap�tulo 1
%
%---------------------------------------------------------------------

\chapter{Introduction}

\begin{resumen}
In this chapter we outline the thesis, state what our objective is and what we hope to achieve and list the programming libraries, techniques and methods used to achieve our goal.
\end{resumen}


%-------------------------------------------------------------------
\section{Overview}
%-------------------------------------------------------------------
\label{cap1:sec:overview}

The logical structure chosen for the program reflects the need for our tool to be a fully functional agent in and out of itself. We have designed it with a clear divider between a backend capable of storing the information and handling at the lowest possible level, which provides the frontend side with easy methods to get the information it needs, which is then processed taking who is going to look at it into account and then adequately presented to the user.
A cornerstone of the program's functionality is to be able to remember decisions taken by a certain user and to be able to compare new data to old data of its kind.
From these two necessities it is natural to consider some kind of identification system for our datasets, as automated as possible so it needs minimum user input and remains independent of the use case.
For our program, if two datasets contain the exact same set of column names then they are considered to be comparable to each other, and every information stored about this kind of datasets will carry an identifier with the column names.
From here onwards, the term 'domain' shall refer to information coming from the same kind of dataset.

%-------------------------------------------------------------------
\section{Workflow}
%-------------------------------------------------------------------
\label{cap1:sec:workflow}

When a new dataset doesn't match any previous knowledge, our program automatically creates a new representation for these datasets which is stored along the others. If it detects a matching JSON with knowledge o its domain it loads that instead.
Each representation of a domain stores data such as how many datasets have been loaded and a number of stats for each dataset and its columns depending on its types which will be specified later.
Also, each domain has a number of 'profiles' associated which correspond to *who* this data is associated with. These profiles contain both historical data of the specific owner of the data (in our practical example, the patient data), and who will watch the report generated by this program, that is, the user of the program.
The information that we're using will be stored in a specific JSON format for each kind that will be specified in chapter 2.
When a dataset is introduced, the program loads the previous information, analyzes it, compares it and generates relevant information to the user. Then it updates the information with both the results of the analysis and user provided information.
This workflow will be the basic use case of the program for every kind of data.

%-------------------------------------------------------------------
\section{Structure}
%-------------------------------------------------------------------
\label{cap1:sec:structure}
A clear module structure is provided so each module does a task in the workflow.
The main modules on the backend side of our application are the Storage module, the CBRStorage module and the Analysis module.
For the frontend, the logic structure will be split into the Reporter module and the Presentation module.

%-------------------------------------------------------------------
\section{Tools}
%-------------------------------------------------------------------
\label{cap1:sec:tools}

Our programming language of choice has been Python, particulary making use of its class to dictionary representation methods which make the work of manipulating the JSON structures much easier than using more rigid languages.
A public repository has been created at (TODO:link here), and we have used Jupyter Notebooks for the testing and formation o a prototype which has been then moved into standard Python packages.

%-------------------------------------------------------------------
\section*{\NotasBibliograficas}
%-------------------------------------------------------------------
\TocNotasBibliograficas

These are the bibliographical notes
\citep{ldesc2e}

\medskip

%Y tambi�n ponemos el acr�nimo \ac{CVS} para que no cruja.


%-------------------------------------------------------------------
\section*{\ProximoCapitulo}
%-------------------------------------------------------------------
\TocProximoCapitulo

This is the next chapter section

% Variable local para emacs, para  que encuentre el fichero maestro de
% compilaci�n y funcionen mejor algunas teclas r�pidas de AucTeX
%%%
%%% Local Variables:
%%% mode: latex
%%% TeX-master: "../Tesis.tex"
%%% End:
