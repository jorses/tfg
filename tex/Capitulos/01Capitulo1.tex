%---------------------------------------------------------------------
%
%                          Cap�tulo 1
%
%---------------------------------------------------------------------

\chapter{Introduction}

\begin{resumen}
In this chapter we outline the thesis, state what is our objective and what we hope to achieve and provide a clear motivation and list the programming libraries, techniques and methods used to achieve our goal.
\end{resumen}


%-------------------------------------------------------------------
\section{Overview}
%-------------------------------------------------------------------
\label{cap1:sec:overview}
%explain logical program structure here. abstract of cap 2

%-------------------------------------------------------------------
\section{Objective}
%-------------------------------------------------------------------
\label{cap1:sec:objective}

We will begin by going over 

%-------------------------------------------------------------------
\section{Method}
%-------------------------------------------------------------------
\label{cap1:sec:method}

We will begin by going over 

%-------------------------------------------------------------------
\section*{\NotasBibliograficas}
%-------------------------------------------------------------------
\TocNotasBibliograficas

These are the bibliographical notes
\citep{ldesc2e}

\medskip

%Y tambi�n ponemos el acr�nimo \ac{CVS} para que no cruja.


%-------------------------------------------------------------------
\section*{\ProximoCapitulo}
%-------------------------------------------------------------------
\TocProximoCapitulo

This is the next chapter section

% Variable local para emacs, para  que encuentre el fichero maestro de
% compilaci�n y funcionen mejor algunas teclas r�pidas de AucTeX
%%%
%%% Local Variables:
%%% mode: latex
%%% TeX-master: "../Tesis.tex"
%%% End:
