%---------------------------------------------------------------------
%
% Cap�tulo 2
%
%---------------------------------------------------------------------

\chapter{CBR : Concept}

\begin{resumen}
In this chapter we outline and explain the user side of the Case Based Reasoning methodology used to test our tool with humans and to better define its ideal workflow.

We provide both an abstract approach and the decisions taken to make the implementation, while leaving the implementation details for the next chapter.
\end{resumen}


%-------------------------------------------------------------------
\section{CBR Design}
%-------------------------------------------------------------------
\label{cap1:sec:introduccion}

Let's first talk about what we mean when we talk about a CBR system.

CBR, or Case-Based Reasoning, is the process of solving new problems based on the solutions of similar problems we may have encountered before.

We can see it as a type of analogy solution making.

It doesn't need an explicit domain model and so it becomes a task of gathering case histories. 

We achieve the reduction of the implementation to essentially identifying significant features that define a case, which is in essence a lot easier than creating a model explicitily. 

CBR systems basically learn by acquiring new knowledge as cases, which combined with data handling techniques and big data make maintaining large volumes of information easier.

\section{CBR Process}
%-------------------------------------------------------------------
\label{cap1:sec:process}

Case-based reasoning can be formulated for a program to emulate as the process that follows:

\begin{enumerate}
\item Retrieve: When facing a new problem, get cases relevant to it from memory. A case is problem, solution, and, optionally, annotations about how the solution was derived. 
\item Reuse: Map the solution from the previous case to the target problem. 
\item Revise: Having mapped the previous solution to the target situation, test the new solution in the real world (or a simulation) and, if necessary, revise. 
\item Retain: After the solution has been successfully adapted to the target problem, store the resulting experience as a new case in memory. 
\end{enumerate}

\section{Retrieve}
%-------------------------------------------------------------------
\label{cap1:sec:retrieve}

Conceptually, we need to get the necessary information about past cases of how they were resolved, that is, mainly what problem it was and how it was solved.
We start from the idea that our problem is basically analyzing a new dataset.
To do this we provide the frame of domains, which ensures us that what we retrieve was relevant condensed information about the problem in the past, and within that information we have the metrics used to analyze datasets like our current one.
Another side of how to solve it is represented by the profile information, which tells us how to solve it for the specific user who is using the program.

\section{Reuse}
%-------------------------------------------------------------------
\label{cap1:sec:reuse}
The knowledge base is obtained primarily from the enumeration of certain past cases or problems. This is built from the fact that experts (humans) are much better at recalling previous experiences and problems than at creating systems of rules. 

As new problems are fed to our expert system (containing the knowledge or memory of previous experiences) to which no past problem can match exactly, the system is capable of reasoning from more general similarities to come up with an answer. 

This tries to imitate the generalization capability of humans.

To map the solution from the previous cases to the current problem what we do is run an analysis on the current dataset, and then use the metrics contained in the domain information, and then apply the profile information on the analysis to filter the results.

\section{Revise}
%-------------------------------------------------------------------
\label{cap1:sec:revise}
When the users are presented with the new report, they have the ability to modify its contents, both graphical and the information they're presented with. This changes are stored in the profile information, so we have access to the new preferences, thus making a better solution from the old one in the spirit of the revise process.

What we're trying to achieve is giving the user as much power as possible, because that is what will make his or her experience better with each time they use the program, which is the whole point behind the implementation of the CBR based system that we've chosen.

\section{Retain}
%-------------------------------------------------------------------
\label{cap1:sec:retain}
The system is able to retain new information by being able to make changes to the elements it stores, namely updating the domain objective analysis with the new analysis and updating the subjective profile report changes with the changes introduced by the users.

This is done only once, after the user has made all the changes and exits the program. No interaction is needed on the side of the user to do this, it is done by default so as to make the whole experience as streamlined as possible.

The memory system present in our system grows and changes by each time we present it with a new case. An important aspect of this memory-based process of reasoning is closely related to automatic learning: our system should be able to remember the problems that it has been presented with and to use past information to solve new challenges.

This is intended to complete our modeling of the human behaviour of CBR, and represent the final step of our CBR system.

