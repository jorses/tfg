%---------------------------------------------------------------------
%
%-----------------------------resumen.tex%----------------------------
%
%---------------------------------------------------------------------
%
% Contiene el cap�tulo del resumen.
%
% Se crea como un cap�tulo sin numeraci�n.
%
%---------------------------------------------------------------------

\chapter{Abstract}
\cabeceraEspecial{Abstract}

\begin{FraseCelebre}
\begin{Frase}
There is light at the end of the tunnel... hopefully it's not a freight train.
\end{Frase}
\begin{Fuente}
M. Carey
\end{Fuente}
\end{FraseCelebre}
\providecommand{\keywords}[1]{\textbf{\textit{Index terms---}} #1}
\keywords{CBR, data visualization, report, IA, artificial intelligence}


This document reflects my Bachelor's Thesis corresponding to the Double Degree in Mathematics and Computer Science, developed within the area of intelligent data analytics and 'Case Based Reasoning'. 
During the progress of the project, the principles applicable in any environment of data processing and the science behind it are explained generally and aimed to be usable in any kind of context by any user provided the right format of data.
Nowadays, highly heterogeneous data collection and processing methods are employed in all industries, 
however the techniques employed to get useful information out of the data usually have a generalistic aim, 
and the work relevant to the field itself is often done manually. In this work we aim to provide an automated way to analyze information while taking into account information and techniques relevant to the field of the analysis.
The objective of this Degree's Final Project is the development of a prototype capable of carrying this analysis while being able to learn based on user input. As a Proof of Concept, we have included several medical domains with each one having developed specific methods and techniques for them.
To serve as a base for this analysis, we have also developed a system for storing, loading and analyzing the information of the domain and the information provided by the user. This system will be the backbone of our architecture and enable the Case Based Reasoning analysis to function correctly in very different situations, providing the metrics and functions needed for every case.

% Variable local para emacs, para  que encuentre el fichero maestro de
% compilaci�n y funcionen mejor algunas teclas r�pidas de AucTeX
%%%
%%% Local Variables:
%%% mode: latex
%%% TeX-master: "../Tesis.tex"
%%% End:
